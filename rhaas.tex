\subsection{Compactifications}



\subsubsection{Admissible Levi decompositions of $P$}

We let $\G_P = \G \cap P$ and $\G_N = \G_P \cap N$. Then the quotient $\G_P/\G_N$ is a non-trivial arithmetic subgroup of $\underline{P}/\underline{N}$ and lies inside the connected component of the identity of the real points of $\underline{P}/\underline{N}$. Furthermore, $\G_P/\G_N$ acts as isometries of spinor norm $1$ on the anisotropic quadratic space $\ell^{\perp}/\ell$  of signature $(1,1)$. Hence $\G_P/\G_N \simeq \Z$ is infinite cyclic. Therefore the exact sequence 
\begin{equation*} \label{exactsequence}
1 \to \G_N \to \G_P \to \G_P/\G_N \to 1
\end{equation*}
splits. We fix $g \in \G_P$ such that its image $\bar{g}$ generates $\G_P/\G_N$. Then $g$ defines a Levi subgroup $M$. In fact, the element $g$ generates $\G_M :=\G_P \cap M$. Hence
\[
\G_P = \G_M \ltimes \G_N.
\]
We will say a Levi decomposition $P = NAM$ is admissible if $
\G_P = (M \cap \G_P) \ltimes \G_N$. In the following we assume that we have picked an admissible Levi decomposition for each rational parabolic.
\\[12pt] 
\textbf{2.2.2 Borel-Serre compactification} 
\\[10pt]
IWe let $\overline{D}$ be the (rational) Borel-Serre enlargement of $D$, see \cite{BorelSerre} or \cite{BJ}, III.9. For any parabolic $\underline{P}$ as before with admissible Levi decomposition $P=NAM$, we define the boundary component
\begin{equation*}
e({P}) = MN \simeq D_W \times W.
\end{equation*}
Here $D_W \simeq M \simeq \R$ is the symmetric space associated to the orthogonal group of $W$. Then $\overline{D}$ is given by 
\begin{equation*}
\overline{D} = D \cup \coprod_{\underline{P}} e({P}),
\end{equation*}
where $\underline{P}$ varies over all rational parabolics. The action of $\G$ on $D$ extends to $\overline{D}$ in a natural way, and we let 
\begin{equation*}
\overline{X}:= \G \back \overline{D}
\end{equation*}
be the Borel-Serre compactification of $X = \G \back D$. This makes $\overline{X}$ a manifold with boundary such that
\[
\partial \overline{X} = \coprod_{[\underline{P}]} e'({P}),
\]
where for each cusp, the corresponding boundary component is given by
\[
e'(P) = \G_P \back e(P).
\]
Here $[\underline{P}]$ runs over all $\G$-conjugacy classes. The space $X_W := \G_M \back D_W$ is a circle. Hence $e'(P)$ is a torus bundle over the circle, where the torus $T^2$ is given by $\G_N \back N$. That is, $
e'(P) = X_W \times T^2$, and we have the natural map $\kappa: e'(P) \to X_W$. We have a natural product neighborhood of $e(P)$ in $\overline{D}$ and hence for $e'(P)$ in $\overline{X}$ given by $[(T,\infty] \times e'(P)]$ for $T$ sufficiently large given by $z(t,s,w)$ with $t>T$. We let $i: X \hookrightarrow \overline{X}$ and $i_P: e'(P)\hookrightarrow \overline{X} $ be the natural inclusions.

It is one of the fundamental properties of the Borel-Serre compactification $\overline{X}$ that it is homotopic equivalent to $X$ itself. Hence their (co)homology groups coincide. 

