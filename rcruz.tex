Note that in view of \eqref{mappingconelift} the lift of classes of $H_2({X}, \partial {X})$ or $H^2(X)$ is the sum of two in general non-holomorphic modular forms (see below).

In \cite{FMres} we systematically study for $\Orth(p,q)$ the restriction of the classes $\theta_{\varphi^V_{q}}$ (also with non-trivial coefficients) to the Borel-Serre boundary. Whenever the restriction vanishes cohomologically, we can expect that a similar analysis to the one given in this paper will give analogous extensions of the geometric theta correspondence. In fact, aside from this paper we have at present managed to do this for several other cases, namely for modular curves with non-trivial coefficients \cite{FMspec} generalizing work of Shintani \cite{Shintani} and for Picard modular surfaces \cite{FM-Cogdell} generalizing work of Cogdell \cite{Cogdell}. 



\subsubsection*{Linking numbers in $3$-manifolds of type Sol}

The theta series $\theta_{\phi_1^W}$ at the boundary is of independent interest and has geometric meaning in its own right. Recall that for two disjoint (rationally) homological trivial $1$-cycles $a$ and $b$ in a $3$-manifold $M$ we can define the {\it linking number} of $a$ and $b$ as the intersection number 
\[
\Lk(a,b) = A \cdot b
\]
of (rational) chains in $M$. Here $A$ is a $2$-chain in $M$ with boundary $a$. We show 

\begin{theorem}\label{FM-linking} (Theorem~\ref{xi'-integralP})
Let $c$ be \bf{homologically} trivial $1$-cycle in $\partial \overline{X}$ which
is disjoint from the torus fibers containing components of $\partial C_n$. Then the holomorphic part of the weight $2$ non-holomorphic modular form $\int_c \theta_{\phi_1^W}$ is given by the generating series of the linking numbers $\sum_{n>0}\Lk(\partial C_n,c) q^n$.
\end{theorem}