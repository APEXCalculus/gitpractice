\subsubsection{Hartman's less than smooth compactification}\label{H-compact}

\vskip 0.3in 

We let $X'$ be the Baily-Borel compactification of $X$, then we ignore it forever. I'm pretty sure they just made this all up.


\vskip 0.3in 

\section{(Co)homology}

\vskip 0.3in 

In this section we describe the relationship between the (co)homology of the various
compactifications. 

\vskip 0.3in 

\subsection{The homology of the boundary components}\label{boundaryhom}

\vskip 0.3in 

Every element of $\Gamma_N =\pi_1(T^2)$ is a rational multiple of
a commutator in $\Gamma_P$ and accordingly the image of $H_1(T^2,\Q)$
in $H_1(e'(P),\Q)$ is trivial.
Let $a_P \in H_1(e'(P),\Z)$ be the class of the identity section
of $\kappa:e'(P) \to X_W$ and $b_P \in H_2(e'(P),\Z)$ be the class
of the torus fiber of $\kappa$. It is clear that the intersection
number of $a_P$ and $b_P$ is $1$ (up to sign) whence $a_P$ and $b_P$
are nontrivial primitive classes. Furthermore, $a_P$ generates
$H_1(e'(P),\Q)$ and $ H_2(e'(P),\Z)  \cong \Z$, generated by $b_P$.
So

\vskip 0.3in 

\begin{lemma}\label{ePhomology}

\begin{enumerate}
\item[(i)] The first rational homology group of $e'(P)$ is generated by $a_P$.
\item[(ii)] The second homology group of $e'(P)$ is generated by $b_P$.
\end{enumerate}
\end{lemma}
\begin{remark} To compute the homology over $\Z$ one has only to use the Wang sequence for a fiber bundle over a circle, see \cite{Milnor}, page 67.
\end{remark}

Let $\Omega_P$ be the unique $P$-invariant $2$-form on $e'(P)$ such that
\begin{equation}\label{areaform}
\int_{b_P} \Omega_P = 1.
\end{equation}
Since $b_P$ is the image of the fundamental class of $T^2$ inside $H_2(e'(P),\Z)$, we see that that the restriction of $\Omega_P$ to $T^2$ lifts to the area form on $W_{\R} \simeq N$ normalized such that $T^2=\G_N \back N$ has area $1$.


\vskip 0.3in 





\subsection{Homology and cohomology of $X$ and $\tilde{X}$}

Accordingly to the discussion in Section~\ref{H-compact} we have the Mayer-Vietoris sequence
\[
0 \to \oplus_P H_2(e'(P)) \to H_2(X) \oplus (\oplus_P S_P)  \to H_2(\tilde{X}) \to 0.
\]
Here $S_{P}$ denotes the span of the classes defined by compactifying divisors at the cusp associated to $P$. The zero on the left comes from $H_3(\tilde{X}) =0$ and the zero on the right comes from the fact that for each $P$ the class $a_P$ injects into $H_1(X^{out})$, see \cite{vGeer}, II.3. Since the generator $b_P$ has trivial intersection with each of the compactifying divisors, $b_P$ bounds on the outside so a fortiori it bounds in $\tilde{X}$. Thus the above short exact sequence is the sum
of the two short exact sequences $\oplus_P H_2(e'(P)) \to H_2(X) \to j_{\ast} H_2(X)$ and
$ 0 \to \oplus_P S_P \to \oplus_P S_P $.  By adding the third terms of the two sequences and equating them to $H_2(\tilde{X})$  we obtain the orthogonal splittings (for the intersection pairing) - see also \cite{vGeer}, p.123,
\begin{align*}\label{vdG2}
H_2(\tilde{X}) = j_{\ast} H_2(X) \oplus \left( \oplus_{[P]} S_{P} \right), \qquad \qquad 
H^2(\tilde{X})  = j_{\#} H_c^2(X) \oplus \left(\oplus_{[P]} S^{\vee}_{P}\right).
\end{align*}
Here $j_{\#}$ is the push-forward map. Furthermore, the pairings on each summand are non-degenerate. Considering $\oplus_P H_2(e'(P)) \to H_2(X) \to j_{\ast} H_2(X)$ we also obtain
\begin{proposition}\label{intersectionhom}
$H_2(\partial \overline{X})$ is the kernel of $j_{\ast}$ so that
\[
j_{\ast} H_2(X) \simeq H_2(X)/ \sum_{[P]} H_2(e'(P)).
\]
\end{proposition}



