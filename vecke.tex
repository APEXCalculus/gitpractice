We recover this feature of the original Hirzebruch-Zagier proof via
\eqref{mappingconelift} with $C=C_m$. The first term on the right
hand side of \eqref{mappingconelift} was studied in the thesis of
the first author of this paper \cite{FCompo} and gives the interior
intersections $(T_n \cdot T_m)_X$ encoded in $F_X$. So via
Theorem~\ref{FMHZ-main} the second term on the right hand side of
\eqref{mappingconelift} must match the boundary contribution
$F_{\infty}$ in \cite{HZ}, that is, we obtain
\begin{theorem}
\[
({T}_n \cdot {T}_m)_{\infty} = \Lk( \partial C_n, \partial C_m).
\]
\end{theorem}
Hence we give an interpretation for the boundary contribution in
\cite{HZ} in terms of linking numbers in $\partial \overline{X}$.
In fact, the construction of $\theta_{\phi_1^W}$ owes a great deal
to Section~2.3 in \cite{HZ}, where a scalar-valued version of
$\theta_{\phi_1^W}$ is introduced, see also Example~\ref{HZbeta}.
Using Theorem~\ref{LinkCnCm} one can also make the connection between
our linking numbers and the formulas of the boundary contribution
in \cite{HZ} explicit.

To summarize, we start with the difference of theta integrals
\eqref{mappingconelift} (which we know a priori is a holomorphic
modular form), then by functorial differential topological computations
we relate its Fourier coefficients to intersection/linking numbers,
and by direct computation of the integrals involved we obtain the
explicit formulas of Hirzebruch-Zagier and a ``closed form'' for
their generating function.

Note that Bruinier \cite{B-123} and Oda \cite{Oda} use related theta
series to consider \cite{HZ}, but their overall approach is different.


\subsubsection*{Currents}

One of the key properties of the cocycle $\varphi^V_2$ is that \\
the $n$-th Fourier coefficients of $\theta_{\varphi^V_{2}}$ \\
represents the Poincar\' e dual class for the cycle $C_n$. \\
Kudla-Millson establish this by showing that $\varphi^V_2$ \\
gives rise to a Thom form for the normal bundle of each of \\
the components of $C_n$. To prove our main result, \\
Theorem~\ref{FMHZ-main}, we follow a different approach \\
using currents which is implicit in \cite{BFDuke} and is \\
closely related to the Green's function $\Xi(n)$ for the \\
divisors $C_n$ constructed by Kudla \cite{KAnn97,KBforms}. \\
This function plays an important role in the Kudla program \\
(see eg \cite{Kmsri}) which considers the analogous generating \\
series for the special cycles in arithmetic geometry. In \\
the non-compact situation however, one needs to modify  \\
$\Xi(n)$ to obtain a Green's function for the cycle $T_n^c$ \\
in $\tilde{X}$. Discussions with U. K\"uhn suggest that the \\
constructions in this paper indeed give rise to such a \\
modification of $\Xi(n)$, see Remark~\ref{Kudla-modification}.\\

\vskip.5cm



We would like to thank Rolf Berndt, Jan Bruinier, Jose Burgos, Misha Kapovich, and Ulf K\"uhn for fruitful and extensive discussions on the constructions and results of this paper. As always it is a great pleasure to thank Steve Kudla for his interest and encouragement. Each of us began the work of relating theta lifts and special cycles with him.

We dedicate this paper to the memory of Gretchen Taylor Millson, beloved wife of the second author.

