\subsection{Rationality of the cap}\label{rat-cap11}
We will now prove Proposition \ref{rat-cap}.  In fact we will show that it holds for any circle $\alpha$ contained in a torus fiber of $e(P)$ and passing through a rational point. We would like to thank Misha Kapovich for simplifying  our original argument. The idea is to construct, for each component of $\partial C_x$, a $2$-chain $A$ with that component as boundary so that $A$ is a sum $P+ T  +\mathcal{M}(\gamma_0)$ of three simplicial $2$-chains in $M$. We then verify that the "parallelogram" $P$  and the "triangle" $T$  have rational area and the period of $\Omega$ over the "monodromy chain" $\mathcal{M}(\gamma_0)$ is zero. 

In what follows we will pass from pictures in the plane involving directed line segments, triangles and parallelograms to identities in the space of simplicial $1$-chains $C_1(T^2)$  on $T^2$. The principal behind this is that any $k$-dimensional subcomplex $S$ of a simplicial complex $Y$ which is the fundamental cycle of an oriented $k$-submanifold $|S|$ (possibly with boundary) of $Y$ corresponds in a {\it unique} way to a sum of oriented $k$-simplices in $C_k(Y)$.



In this subsection we will work with a general $3$-manifold $M$ with Sol geometry. Of course this includes all the manifolds $e'(P)$ that occur in this paper. Let $f \in \SL(2,\Z)$ be a hyperbolic element. We will then consider the $3$-manifold $M$ obtained from $ \R \times T^2$ (with the $2$-torus $T^2 = W/ \Z^2$) given by the relation
\begin{equation}\label{glueing}
(s,w) \sim (s+1,f(w)).  
\end{equation}
We let $\pi: \R\times T^2 \to M$ be the resulting infinite cyclic covering. 



We now define notation we will use below. We will use Greek letters to denote closed geodesics on $T^2$, a subscript $c$  will indicate that the geodesic starts at the point $c$ on $T^2$. We will use the analogous notation for geodesic arcs on $W$.  We will use $[\alpha]$ 
to denote the corresponding homology class of a closed geodesic $\alpha$ on $T^2$. 
If $x$ and $y$ are points on $W$ we will use $\overline{xy}$ to denote the oriented line segment joining $x$ to $y$ and $\overrightarrow{xy}$ to denote the corresponding (free) vector
i.e. the equivalence class of $\overline{xy}$ modulo parallel translation. 
   
We first take care of the fact that $\alpha$ does not necessarily pass through the origin. For convenience we will assume $\alpha$ is in the fiber over the base-point $z(x)$ corresponding to $s=0$. Let $\alpha_0$ be the parallel translate of $\alpha$ to the origin. Then we can find a cylinder $P$, image of an oriented parallelogram $\widetilde{P}$ under the universal cover $W \to T^2$ with rational vertices, such that in $Z_1(T^2,\Q)$, the group of rational $1$-cycles, we have
\begin{equation}\label{firstrectangle}
\partial P = \alpha - \alpha_0.
\end{equation}
Since $\widetilde{P}$ has rational vertices we find $\int_{P} \Omega = \int_{\widetilde{P}} \Omega \in \Q$.

Now we take care of the harder part of finding $A$ as above. The key is the construction  of "monodromy $2$-chains".  For any closed geodesic $\gamma_0 \subset T^2$ starting at $0$ we define the monodromy $2$-chain  $\mathcal{M}(\gamma_0)$ to be the image of the cylinder $\gamma_0 \times [0,1] \subset T^2 \times \R$ in $M$.
The reader will verify using \eqref{glueing} that in $Z_1(T^2,\Q)$ we have 
\begin{equation} \label{boundaryofmon}
 \partial \mathcal{M}(\gamma_0) = f^{-1}(\gamma_0) -\gamma_0.
\end{equation}
Since $f$ preserves the origin, the geodesic $f^{-1}(\gamma_0)$ is also a closed geodesic starting at the origin. Since $f^{-1}$ is hyperbolic we have $|\tr(f^{-1})| >2$ and hence $\det(f^{-1} -I)=  det( I - f) = \tr(f) -2 \neq 0$. Put $N= \det(f^{-1} -I)$ and define $[\gamma_0] \in H_1(T^2,\Z)$ by  
\begin{equation}\label{invertmatrix}
f^{-1}([\gamma_0]) -[\gamma_0] = N[\alpha_0]. 
\end{equation}
Note that $[\gamma_0] = N \{(f^{-1} - I)^{-1} ([\alpha_0] \}$ is necessarily an integer homology class. Also note that is an equation in the first {\it homology}, it is not an equation in the group of $1$-cycles $Z_1(T^2,\Q)$. Since any homology class contains a unique closed geodesic starting at the origin we obtain a closed geodesic $\gamma_0 \in [\gamma_0]$  and a corresponding  monodromy $2$-chain $\mathcal{M}(\gamma_0)$ whence \eqref{boundaryofmon} holds in $Z_1(T^2,\Q)$. We now solve

\begin{problem}
Find an equation in $Z_1(T^2,\Z)$ which descends to  \eqref{invertmatrix}. 
\end{problem}
