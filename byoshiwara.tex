We also have a crucial restriction result for the singular form ${ \tilde{\psi}_{0,1}}$. However, one needs to be careful in forming the naive theta series associated to $\tilde{\psi}_{0,1}$ by summing over all (non-zero) lattice elements. This would give a form on $X$ with singularities on a dense subset of $X$. Instead we define $\tilde{\psi}_{{2,0}}(n)$ in the same way as for $\varphi_2(n)$ by summing over all non-zero $x \in \calL_V$ of length $n \in \Q$. This gives a $1$-form on $X$ which for $n>0$ has singularities along the locally finite cycle $C_n$. Similarly, we define
\[
\tilde{\psi}_{0,1}^P(n)= \sum_{\substack{x \in \calL_{W_P}, (x,x)=2n}} \tilde{\psi}_{0,1}^P(x),
\]
\vskip-.2cm
\noindent
which descends to a $1$-form on $e'(P)$ with singularities. We also define $\tilde{\psi'}_{0,1}(n)$ 
and $\phi_{0,1}^P(n)$ in the same way. We have

\begin{proposition}\label{psitilderes}
The restriction of the $1$-form $\tilde{\psi}_{{1}}(n)$ to $e'(P)$ is given by 
\[
i_P^{\ast} \tilde{\psi}_1(n) = \tilde{\psi}_{0,1}^P(n).
\]
\end{proposition}

\begin{proof}
We assume that $P$ is the stabilizer of the isotropic line $\ell=\Q u$. For $x = au + x_W + bu'$, we have for the majorant at $z=(w,t,s)$ the formula
\[
(x,x)_z = \frac1{t^2}(a-(x_W,w)-bq(w))^2 + (x_w+bw,x_w+bw)_s +b^2t^2.
\]
Here $(\,,\,)_s$ is the majorant associated to $W$. Hence by \eqref{GreeneqV} and \eqref{psi20} we see that the sum of all $x \in \calL_V$ with $b \ne 0$ in $\tilde{\psi}_1(n)$ is uniformly rapidly decreasing as $t \to \infty$. Now fix an element $x_W \in \calL_W$. Then  $x_W +(a+h)u \in \calL_V$ for all $a \in \Z$ for some $h \in \Q/\Z$; in fact all elements in $\calL_V \cap u^{\perp}$ are of this form. We consider $\sum_{a \in \Z} \tilde{\psi}_1(x_W +(a+h)u,z)$ as $t \to \infty$. By considerations as in \cite{FMres}, sections 4 and 9, we can assume $w=0$ and $s=0$. We apply Poisson summation for the sum on $a \in \Z$ and obtain
\begin{align*}
\sum_{a \in \Z} \tilde{\psi}_1(x_W +au,z) = \sum_{k \in \Z} \left( \int_{1}^{\infty} P(x,t,r) e^{-2\pi x_3^2r +t^2k^2/r} \frac{dr}{r} \right) e^{-2\pi i k h} e^{-\pi (x_W,x_W)},
\end{align*}
where
\[
P(x,t,r) = \frac{x_2x_3\sqrt{r}}{\sqrt{2}} dw_2 + \frac{1}{2\sqrt{2}}\left(\frac1{2\pi}-\frac{t^2k^2}{r}\right)dw_3 - \frac{i x_3 k}{\sqrt{2}}dt +  \frac{i x_2kt}{\sqrt{2}} ds.\]
Now the sum over all $k \ne 0$ is rapidly decreasing while for $k=0$ we obtain $\tilde{\psi}_{0,1}(x_W)$. If $x_W=0$, i.e., for $n=0$ one needs to argue slightly differently. Then we have
\[
\sum_{a \ne 0} \tilde{\psi}_1(au,z) = \frac{1}{2\sqrt{2}\pi} \sum_{a \ne 0} e^{-\pi a^2/t^2} \frac{dw_3}{t} =  \frac{1}{2\sqrt{2}\pi} \left(\sum_{k \in \Z} e^{-\pi t^2k^2} \right) dw_3 -  \frac{1}{2\sqrt{2}\pi} \frac{dw_3}{t},
\]
which goes to $ \tfrac{1}{2\sqrt{2}\pi} dw_3 = \tilde{\psi}_{0,1}(0)$. 
This proves the proposition.
\end{proof}




\subsection{Main result}

In the previous sections, we constructed a closed $2$-form \\
$\theta_{\varphi_2}$ on $\overline{X}$ such that the restriction\\
of $\theta_{\varphi_2}$ to the boundary $\partial \overline{X}$\\
was exact with primitive $ \sum_{[\underline{P}]} \theta^P_{\phi_{0,1}}$. \\
From now on we usually write $\varphi$ for $\varphi_2$ and $\phi$ for \\
$\phi_{0,1}$ if it does not cause any confusion. By the definition of \\
the differential for the mapping cone complex $C^{\bullet}$ we \\
immediately obtain by Theorem~\ref{restriction} and Theorem~\ref{globalexact}\\
\begin{proposition}
The pair $(\theta_{\varphi_2}(\calL_V), \sum_{[P]} \theta^P_{\phi_{0,1}}(\calL_{W_P}))$ is a $2$-cocycle in $C^{\bullet}$.
\end{proposition}

We write for short $(\theta_{\varphi},\theta_{\phi})$. We obtain a class $[[\theta_{\varphi},\theta_{\phi}]]$ in $H^2(C^{\bullet})$ and hence a class $[\theta_{\varphi},\theta_{\phi}]$ in $H^2_c(X)$. The pairing with $[\theta_{\varphi}, \theta_{\phi}]$ then defines a lift $\Lambda^c$ on differential $2$-forms on $\overline{X}$, which factors through $H^2(\overline{X}) = H^2(X)$. By Lemma~\ref{integralformula} it is given by 
\[
\Lambda^c(\eta,\tau) =  \int_{\overline{X}} \eta \wedge \theta_{\varphi_2} - \sum_{[P]}  \int_{e'(P)} i^*\eta \wedge \theta^P_{\phi_{0,1}}.
\]

\begin{theorem}\label{La^c-hol}
The class $[[\theta_{\varphi},  \theta_{\phi}]]$ is holomorphic, that is, 
\[
L\left(\theta_{\varphi}, \theta_{\phi}\right) = d(\theta_{\psi_1},0).
\]
Hence $[\theta_{\varphi}, \theta_{\phi}]$ is a holomorphic modular form with values in the compactly supported cohomology of $X$, so that the lift $\Lambda^c$ takes values in the holomorphic modular forms.
\end{theorem}
\begin{proof}
By Theorem~\ref{restriction} and Theorem~\ref{globalholomorphic2} we calculate 
\[
d(\theta_{\psi_1},0) = (d\theta_{\psi_1}, i^{\ast} \theta_{\psi_1}) = \left(L \theta_{\varphi_2},  \sum_{[P]}\theta^P_{\psi_{0,1}}\right) =L\left(\theta_{\varphi_2}, \sum_{[P]} \theta^P_{\phi_{0,1}}\right). \qedhere
\]
\end{proof}

It remains to compute the Fourier expansion in $\tau$ of $[\theta_{\varphi}, \theta_{\phi}](\tau)$. We will carry this out in Section~\ref{currents}. 

\begin{theorem}\label{FM-main-th}
We have 
\[
[\theta_{\varphi}, \theta_{\phi}](\tau) =  -\frac{1}{2\pi}\delta_{h0} [\omega] + \sum_{n>0} \PD[C^c_n] q^n \in H_c^2(X,\Q) \otimes M_2(\G(N)).
\]
That is, for any closed $2$-form $\eta$ on $\overline{X}$
\[
\Lambda^c(\eta,\tau) = -\frac{1}{2\pi}\delta_{h0} \int_X \eta \wedge \omega + \sum_{n>0} \left( \int_{C^c_n} \eta \right)q^n,
\]
In particular, the map takes values in the holomorphic modular forms and factors through cohomology. We obtain a map
\begin{equation}
\Lambda^c: H^{2}(X) \to M_{2}(\G(N))
\end{equation}
from the cohomology with compact supports to the space of holomorphic modular forms of weight $2$ for the principal congruence subgroup $\G(N) \subseteq \SL_2(\Z))$.
Alternatively, for $C$ any relative $2$-cycle in $X$ defining a class in $H_2(\overline{X},\partial \overline{X},\Z)$, we have
\[
\Lambda^c(C,\tau) = -\frac{1}{2\pi}\delta_{h0} \vol(C) + \sum_{n>0} ( C^c_n \cdot C ) q^n \in M_2(\G(N)).
\]
\end{theorem}

\begin{remark}
In the theorem we now consider the K\"ahler form $\omega$ representing a class in the compactly supported cohomology. In fact, our mapping cone construction gives an explicit coboundary by which $\omega$ is modified to become rapidly decreasing. 
\end{remark}
