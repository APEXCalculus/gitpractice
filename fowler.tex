\begin{comment}

We can also realize $M$ in terms of an isotropic basis $u_2,u_2'$ of  $W_{\R}$ as 
\[
{M} = 
\{ m'(s)=
\begin{pmatrix}
1& &  \\ 
 &    \begin{smallmatrix} s &  \\  & s^{-1}
 \end{smallmatrix}     &  \\
 &         & 1\\
\end{pmatrix}; \, s \in \R_+
 \}, 
\]
so that $m(s) =  m'(e^s)$. Note $M \simeq \SO_0(W_{\R})$. We write $\frak{n},\frak{a},\frak{m}$ for their Lie algebras. 




The $P$-invariant extensions of the pullback of the $\omega_{\alpha\mu}$'s under the natural map $\sigma: \frak{n}\frak{a}\frak{m} \to \frak{g} \to \frak{g}/\frak{k} \simeq \frak{p}$ are given by
\begin{align*}
 \sigma^{\ast} \omega_{13}  = \frac12( - \frac{dw}{s't} + \frac{dw'}{t/s'}), \quad \sigma^{\ast} \omega_{14}  = \frac{dt}{t}, \quad
  \sigma^{\ast} \omega_{24} = \frac12(  \frac{dw}{s't} + \frac{dw'}{t/s'}),  \quad \sigma^{\ast} \omega_{23}= \frac{ds'}{s'}.
\end{align*}
Here we picked coordinates for $W_{\R}$ by setting $w= wu_2+w'u_2'$. 




Of course, it is very well known that $D \simeq \h \times \h$, the product of two upper half planes. To see this in our setting, we choose an isomorphism $V_{\R} \simeq M_2(\R)$ such that $u = \kzxz{1}{0}{0}{0}$ and $u' = \kzxz{0}{0}{0}{1}$ and such that the quadratic form $q(x) = (x,x)/2$ for $ x \in M_2(\R)$ is given by $q(x) = \det(x)$. Note that in this model, we can pick in addition $e_2= \tfrac1{\sqrt{2}}\kzxz{0}{1}{-1}{0}$ and $e_3= \tfrac1{\sqrt{2}}\kzxz{0}{1}{1}{0}$. 
Then $SL_2(\R) \times SL_2(\R)$ acts on $M_2(\R)$ by $(g_1,g_2)x = g_1x\, {^{t}g_2}$ as isometries, and this realizes the isomorphism $\Spin(2,2) \simeq SL_2{\R} \times SL_2(\R)$. Under this isomorphism, we have 
\begin{align*}
a(t) rightarrow (   \kzxz{\sqrt{t}}{}{}{\sqrt{t}^{-1}}, \kzxz{\sqrt{t}}{}{}{\sqrt{t}^{-1}} )&, \qquad \qquad
m'(s) rightarrow  (   \kzxz{\sqrt{s}}{}{}{\sqrt{s}^{-1}}, \kzxz{\sqrt{s}^{-1}}{}{}{\sqrt{s}} ) \\
n(w,w')  &rightarrow (   \kzxz{1}{-w}{}{1}, \kzxz{1}{w'}{}{1} )
\end{align*}
This makes the isomorphism $D \simeq \h \times \h$ explicit. Namely, for $(z_1,z_2)= (x_1+iy_1,x_2+iy_2) \in \h \times \h$ we have
\[
y_1= st ,\qquad y_2=t/s, \qquad  x_1=-w, \qquad x_2 = w'.
\]

\end{comment}

\subsubsection{Arithmetic Quotient}


We let $L$ be an even lattice in $V$ of level $N$, that is $L \subseteq L^{\#}$, the dual lattice, $(x,x) \in 2 \Z$ for $x \in L$, and $q(L^{\#}) \Z = \tfrac1{N}\Z$. We fix $h \in L^{\#}$ and let $\Gamma \subseteq \Stab{L}$ be a subgroup of finite index of the stabilizer of $\mathcal{L}:=L+h$ in $G$. For each isotropic line $\ell =\Q u$, we assume that $u$ is primitive in the lattice $L$ in $V$. We will throughout assume that the $\Q$-rank of $\underline{G}$ is $1$, that is, $V$ splits exactly one hyperbolic plane over $\Q$. Then we define the Hilbert modular surface
\[
X = \G \back D.
\]

\begin{example}\label{HZex}

An important example is the following. Let $d>0$ be the discriminant of the real quadratic field $K = \Q(\sqrt{d})$ over $\Q$, $\mathcal{O}_K$ its ring of integers. We denote by $x \mapsto x'$
the Galois involution on $K$. We let $V \subset M_2(K)$ be the space
of skew-hermitian matrices in $M_2(K)$, i.e., which satisfy $^tx' =-x$. Then the determinant on $M_2(K)$ gives $V$ the structure of a non-degenerate rational quadratic space of signature $(2,2)$ and $\Q$-rank $1$. We define the integral skew-hermitian matrices by 
\begin{equation*}
L = \{ x = ( \begin{smallmatrix} a\sqrt{d}&\lambda\\-\lambda'&b\sqrt{d}
  \end{smallmatrix}  ) \; : \; a,b \in \Z, \; \lambda  \in
  \mathcal{O}_K \}.
\end{equation*}
 Then $L$ is a lattice of level $d$. We embed $\SL_2(K)$ into $\SL_2{\R} \times \SL_2(\R)$ by $g \mapsto (g,g')$ so that $\SL_2(\mathcal{O}_K)$ acts on $L$ by $\g.x = \g x{^t\g'}$ as isometries. Hirzebruch and Zagier actually considered this case for $d \equiv 1 \pmod{4}$ a prime.
\end{example}

The quotient space $X$ is in general an oriented uniformizable orbifold with isolated singularities.  We will treat $X$ as a manifold - we will use Stokes' Theorem and Poincar\'e duality over $\Q$ on $X$. This is justified because in each instance we can pass to a finite normal cover $Y$ of $X$ with $Y$ a manifold. Hence, the formulas we want hold on $Y$. 
We then then go back to the quotient by taking invariants or summing over the group $\Phi$ of covering transformations. The point is that the de Rham complex of $X$ is the algebra of $\Phi$-invariants in the one of $Y$ and the {\it rational} homology (cohomology) groups of $X$ are the groups of $\Phi$-coinvariants (invariants) of those
 of $Y$. 